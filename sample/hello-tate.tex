\documentclass[a5j]{ltjtbook}

\usepackage{luatexja}


\usepackage[ipaex]{luatexja-preset}

% ルビ
\usepackage{luatexja-ruby}

\title{吾輩は猫である}
\author{夏目漱石}
\date{}

\begin{document}


 \ruby{吾輩}{わがはい}は猫である。名前はまだ無い。 \newline
 どこで生れたかとんと\ruby{見当}{けんとう}がつかぬ。何でも薄暗いじめじめした所でニャーニャー泣いていた事だけは記憶している。吾輩はここで始めて人間というものを見た。しかもあとで聞くとそれは書生という人間中で一番\ruby{獰悪}{どうあく}な種族であったそうだ。この書生というのは時々我々を\ruby{捕}{つかま}えて\ruby{煮}{に}て食うという話である。しかしその当時は何という考もなかったから別段恐しいとも思わなかった。ただ彼の\ruby{掌}{てのひら}に載せられてスーと持ち上げられた時何だかフワフワした感じがあったばかりである。掌の上で少し落ちついて書生の顔を見たのがいわゆる人間というものの\ruby{見始}{みはじめ}であろう。この時妙なものだと思った感じが今でも残っている。第一毛をもって装飾されべきはずの顔がつるつるしてまるで\ruby{薬缶}{やかん}だ。その\ruby{後}{ご}猫にもだいぶ\ruby{逢}{あ}ったがこんな\ruby{片輪}{かたわ}には一度も\ruby{出会}{でく}わした事がない。のみならず顔の真中があまりに突起している。そうしてその穴の中から時々ぷうぷうと\ruby{煙}{けむり}を吹く。どうも\ruby{咽}{む}せぽくて実に弱った。これが人間の飲む\ruby{煙草}{たばこ}というものである事はようやくこの頃知った。 \newline
 この書生の掌の\ruby{裏}{うち}でしばらくはよい心持に坐っておったが、しばらくすると非常な速力で運転し始めた。書生が動くのか自分だけが動くのか分らないが\ruby{無暗}{むやみ}に眼が廻る。胸が悪くなる。\ruby{到底}{とうてい}助からないと思っていると、どさりと音がして眼から火が出た。それまでは記憶しているがあとは何の事やらいくら考え出そうとしても分らない。 \newline
 ふと気が付いて見ると書生はいない。たくさんおった兄弟が一\ruby{疋}{ぴき}も見えぬ。\ruby{肝心}{かんじん}の母親さえ姿を隠してしまった。その上\ruby{今}{いま}までの所とは違って\ruby{無暗}{むやみ}に明るい。眼を明いていられぬくらいだ。はてな何でも\ruby{容子}{ようす}がおかしいと、のそのそ\ruby{這}{は}い出して見ると非常に痛い。吾輩は\ruby{藁}{わら}の上から急に笹原の中へ棄てられたのである。 \newline
 ようやくの思いで笹原を這い出すと向うに大きな池がある。吾輩は池の前に坐ってどうしたらよかろうと考えて見た。別にこれという\ruby{分別}{ふんべつ}も出ない。しばらくして泣いたら書生がまた迎に来てくれるかと考え付いた。ニャー、ニャーと試みにやって見たが誰も来ない。そのうち池の上をさらさらと風が渡って日が暮れかかる。腹が非常に減って来た。泣きたくても声が出ない。仕方がない、何でもよいから\ruby{食物}{くいもの}のある所まであるこうと決心をしてそろりそろりと池を\ruby{左}{ひだ}りに廻り始めた。どうも非常に苦しい。そこを我慢して無理やりに\ruby{這}{は}って行くとようやくの事で何となく人間臭い所へ出た。ここへ\ruby{這入}{はい}ったら、どうにかなると思って竹垣の\ruby{崩}{くず}れた穴から、とある邸内にもぐり込んだ。縁は不思議なもので、もしこの竹垣が破れていなかったなら、吾輩はついに\ruby{路傍}{ろぼう}に\ruby{餓死}{がし}したかも知れんのである。一樹の蔭とはよく\ruby{云}{い}ったものだ。この垣根の穴は\ruby{今日}{こんにち}に至るまで吾輩が\ruby{隣家}{となり}の三毛を訪問する時の通路になっている。さて\ruby{邸}{やしき}へは忍び込んだもののこれから先どうして\ruby{善}{い}いか分らない。そのうちに暗くなる、腹は減る、寒さは寒し、雨が降って来るという始末でもう一刻の\ruby{猶予}{ゆうよ}が出来なくなった。仕方がないからとにかく明るくて暖かそうな方へ方へとあるいて行く。今から考えるとその時はすでに家の内に這入っておったのだ。ここで吾輩は\ruby{彼}{か}の書生以外の人間を再び見るべき機会に\ruby{遭遇}{そうぐう}したのである。第一に逢ったのがおさんである。これは前の書生より一層乱暴な方で吾輩を見るや否やいきなり\ruby{頸筋}{くびすじ}をつかんで表へ\ruby{抛}{ほう}り出した。いやこれは駄目だと思ったから眼をねぶって運を天に任せていた。しかしひもじいのと寒いのにはどうしても我慢が出来ん。吾輩は再びおさんの\ruby{隙}{すき}を見て台所へ\ruby{這}{は}い\ruby{上}{あが}った。すると間もなくまた投げ出された。吾輩は投げ出されては這い上り、這い上っては投げ出され、何でも同じ事を四五遍繰り返したのを記憶している。その時におさんと云う者はつくづくいやになった。この間おさんの\ruby{三馬}{さんま}を\ruby{偸}{ぬす}んでこの返報をしてやってから、やっと胸の\ruby{痞}{つかえ}が下りた。吾輩が最後につまみ出されようとしたときに、この\ruby{家}{うち}の主人が騒々しい何だといいながら出て来た。下女は吾輩をぶら下げて主人の方へ向けてこの\ruby{宿}{やど}なしの小猫がいくら出しても出しても\ruby{御台所}{おだいどころ}へ\ruby{上}{あが}って来て困りますという。主人は鼻の下の黒い毛を\ruby{撚}{ひね}りながら吾輩の顔をしばらく\ruby{眺}{なが}めておったが、やがてそんなら内へ置いてやれといったまま奥へ\ruby{這入}{はい}ってしまった。主人はあまり口を聞かぬ人と見えた。下女は\ruby{口惜}{くや}しそうに吾輩を台所へ\ruby{抛}{ほう}り出した。かくして吾輩はついにこの\ruby{家}{うち}を自分の\ruby{住家}{すみか}と\ruby{極}{き}める事にしたのである。 \newline
 吾輩の主人は\ruby{滅多}{めった}に吾輩と顔を合せる事がない。職業は教師だそうだ。学校から帰ると終日書斎に這入ったぎりほとんど出て来る事がない。家のものは大変な勉強家だと思っている。当人も勉強家であるかのごとく見せている。しかし実際はうちのものがいうような勤勉家ではない。吾輩は時々忍び足に彼の書斎を\ruby{覗}{のぞ}いて見るが、彼はよく\ruby{昼寝}{ひるね}をしている事がある。時々読みかけてある本の上に\ruby{涎}{よだれ}をたらしている。彼は胃弱で皮膚の色が\ruby{淡黄色}{たんこうしょく}を帯びて弾力のない\ruby{不活溌}{ふかっぱつ}な徴候をあらわしている。その癖に大飯を食う。大飯を食った\ruby{後}{あと}でタカジヤスターゼを飲む。飲んだ後で書物をひろげる。二三ページ読むと眠くなる。涎を本の上へ垂らす。これが彼の毎夜繰り返す日課である。吾輩は猫ながら時々考える事がある。教師というものは実に\ruby{楽}{らく}なものだ。人間と生れたら教師となるに限る。こんなに寝ていて勤まるものなら猫にでも出来ぬ事はないと。それでも主人に云わせると教師ほどつらいものはないそうで彼は友達が来る\ruby{度}{たび}に何とかかんとか不平を鳴らしている。 \newline
 吾輩がこの家へ住み込んだ当時は、主人以外のものにははなはだ不人望であった。どこへ行っても\ruby{跳}{は}ね付けられて相手にしてくれ手がなかった。いかに珍重されなかったかは、\ruby{今日}{こんにち}に至るまで名前さえつけてくれないのでも分る。吾輩は仕方がないから、出来得る限り吾輩を入れてくれた主人の\ruby{傍}{そば}にいる事をつとめた。朝主人が新聞を読むときは必ず彼の\ruby{膝}{ひざ}の上に乗る。彼が昼寝をするときは必ずその\ruby{背中}{せなか}に乗る。これはあながち主人が好きという訳ではないが別に構い手がなかったからやむを得んのである。その後いろいろ経験の上、朝は\ruby{飯櫃}{めしびつ}の上、夜は\ruby{炬燵}{こたつ}の上、天気のよい昼は\ruby{椽側}{えんがわ}へ寝る事とした。しかし一番心持の好いのは\ruby{夜}{よ}に\ruby{入}{い}ってここのうちの小供の寝床へもぐり込んでいっしょにねる事である。この小供というのは五つと三つで夜になると二人が一つ床へ\ruby{入}{はい}って\ruby{一間}{ひとま}へ寝る。吾輩はいつでも彼等の中間に\ruby{己}{おの}れを\ruby{容}{い}るべき余地を\ruby{見出}{みいだ}してどうにか、こうにか割り込むのであるが、運悪く小供の一人が眼を\ruby{醒}{さ}ますが最後大変な事になる。小供は――ことに小さい方が\ruby{質}{たち}がわるい――猫が来た猫が来たといって夜中でも何でも大きな声で泣き出すのである。すると例の神経胃弱性の主人は\ruby{必}{かなら}ず眼をさまして次の部屋から飛び出してくる。現にせんだってなどは\ruby{物指}{ものさし}で尻ぺたをひどく\ruby{叩}{たた}かれた。 \newline
 吾輩は人間と同居して彼等を観察すればするほど、彼等は\ruby{我儘}{わがまま}なものだと断言せざるを得ないようになった。ことに吾輩が時々\ruby{同衾}{どうきん}する小供のごときに至っては\ruby{言語同断}{ごんごどうだん}である。自分の勝手な時は人を逆さにしたり、頭へ袋をかぶせたり、\ruby{抛}{ほう}り出したり、\ltjkenten{へっつい}の中へ押し込んだりする。しかも吾輩の方で少しでも手出しをしようものなら\ruby{家内}{かない}総がかりで追い廻して迫害を加える。この間もちょっと畳で爪を\ruby{磨}{と}いだら細君が非常に\ruby{怒}{おこ}ってそれから容易に座敷へ\ruby{入}{い}れない。台所の板の間で\ruby{他}{ひと}が\ruby{顫}{ふる}えていても\ruby{一向}{いっこう}平気なものである。吾輩の尊敬する\ruby{筋向}{すじむこう}の白君などは\ruby{逢}{あ}う\ruby{度毎}{たびごと}に人間ほど不人情なものはないと言っておらるる。白君は先日玉のような子猫を四疋\ruby{産}{う}まれたのである。ところがそこの\ruby{家}{うち}の書生が三日目にそいつを裏の池へ持って行って四疋ながら棄てて来たそうだ。白君は涙を流してその一部始終を話した上、どうしても我等\ruby{猫族}{ねこぞく}が親子の愛を\ruby{完}{まった}くして美しい家族的生活をするには人間と戦ってこれを\ruby{剿滅}{そうめつ}せねばならぬといわれた。一々もっともの議論と思う。また隣りの\ruby{三毛}{みけ}君などは人間が所有権という事を解していないといって\ruby{大}{おおい}に憤慨している。元来我々同族間では\ruby{目刺}{めざし}の頭でも\ruby{鰡}{ぼら}の\ruby{臍}{へそ}でも一番先に見付けたものがこれを食う権利があるものとなっている。もし相手がこの規約を守らなければ腕力に訴えて\ruby{善}{よ}いくらいのものだ。しかるに彼等人間は\ruby{毫}{ごう}もこの観念がないと見えて我等が見付けた御馳走は必ず彼等のために\ruby{掠奪}{りゃくだつ}せらるるのである。彼等はその強力を頼んで正当に吾人が食い得べきものを\ruby{奪}{うば}ってすましている。白君は軍人の家におり三毛君は代言の主人を持っている。吾輩は教師の家に住んでいるだけ、こんな事に関すると両君よりもむしろ楽天である。ただその日その日がどうにかこうにか送られればよい。いくら人間だって、そういつまでも栄える事もあるまい。まあ気を永く猫の時節を待つがよかろう。 \newline
 \ruby{我儘}{わがまま}で思い出したからちょっと吾輩の家の主人がこの我儘で失敗した話をしよう。元来この主人は何といって人に\ruby{勝}{すぐ}れて出来る事もないが、何にでもよく手を出したがる。俳句をやって\ltjkenten{ほととぎす}へ投書をしたり、新体詩を\ltjkenten{明星}へ出したり、間違いだらけの英文をかいたり、時によると弓に\ruby{凝}{こ}ったり、\ruby{謡}{うたい}を習ったり、またあるときはヴァイオリンなどをブーブー鳴らしたりするが、気の毒な事には、どれもこれも物になっておらん。その癖やり出すと胃弱の癖にいやに熱心だ。\ruby{後架}{こうか}の中で謡をうたって、近所で\ruby{後架先生}{こうかせんせい}と\ruby{渾名}{あだな}をつけられているにも関せず\ruby{一向}{いっこう}平気なもので、やはりこれは\ruby{平}{たいら}の\ruby{宗盛}{むねもり}にて\ruby{候}{そうろう}を繰返している。みんながそら宗盛だと吹き出すくらいである。この主人がどういう考になったものか吾輩の住み込んでから一月ばかり\ruby{後}{のち}のある月の月給日に、大きな包みを\ruby{提}{さ}げてあわただしく帰って来た。何を買って来たのかと思うと水彩絵具と毛筆とワットマンという紙で今日から謡や俳句をやめて絵をかく決心と見えた。果して翌日から当分の間というものは毎日毎日書斎で昼寝もしないで絵ばかりかいている。しかしそのかき上げたものを見ると何をかいたものやら誰にも鑑定がつかない。当人もあまり\ruby{甘}{うま}くないと思ったものか、ある日その友人で美学とかをやっている人が来た時に\ruby{下}{しも}のような話をしているのを聞いた。 \newline
「どうも\ruby{甘}{うま}くかけないものだね。人のを見ると何でもないようだが\ruby{自}{みずか}ら筆をとって見ると\ruby{今更}{いまさら}のようにむずかしく感ずる」これは主人の\ruby{述懐}{じゅっかい}である。なるほど\ruby{詐}{いつわ}りのない処だ。彼の友は金縁の\ruby{眼鏡越}{めがねごし}に主人の顔を見ながら、「そう初めから上手にはかけないさ、第一室内の想像ばかりで\ruby{画}{え}がかける訳のものではない。\ruby{昔}{むか}し\ruby{以太利}{イタリー}の大家アンドレア・デル・サルトが言った事がある。画をかくなら何でも自然その物を写せ。天に\ruby{星辰}{せいしん}あり。地に\ruby{露華}{ろか}あり。飛ぶに\ruby{禽}{とり}あり。走るに\ruby{獣}{けもの}あり。池に金魚あり。\ruby{枯木}{こぼく}に\ruby{寒鴉}{かんあ}あり。自然はこれ一幅の\ruby{大活画}{だいかつが}なりと。どうだ君も画らしい画をかこうと思うならちと写生をしたら」 \newline
「へえアンドレア・デル・サルトがそんな事をいった事があるかい。ちっとも知らなかった。なるほどこりゃもっともだ。実にその通りだ」と主人は\ruby{無暗}{むやみ}に感心している。金縁の裏には\ruby{嘲}{あざ}けるような\ruby{笑}{わらい}が見えた。 \newline
 その翌日吾輩は例のごとく\ruby{椽側}{えんがわ}に出て心持善く\ruby{昼寝}{ひるね}をしていたら、主人が例になく書斎から出て来て吾輩の\ruby{後}{うし}ろで何かしきりにやっている。ふと眼が\ruby{覚}{さ}めて何をしているかと\ruby{一分}{いちぶ}ばかり細目に眼をあけて見ると、彼は余念もなくアンドレア・デル・サルトを\ruby{極}{き}め込んでいる。吾輩はこの有様を見て覚えず失笑するのを禁じ得なかった。彼は彼の友に\ruby{揶揄}{やゆ}せられたる結果としてまず手初めに吾輩を写生しつつあるのである。吾輩はすでに\ruby{十分}{じゅうぶん}寝た。\ruby{欠伸}{あくび}がしたくてたまらない。しかしせっかく主人が熱心に筆を\ruby{執}{と}っているのを動いては気の毒だと思って、じっと\ruby{辛棒}{しんぼう}しておった。彼は今吾輩の輪廓をかき上げて顔のあたりを\ruby{色彩}{いろど}っている。吾輩は自白する。吾輩は猫として決して上乗の出来ではない。背といい毛並といい顔の造作といいあえて他の猫に\ruby{勝}{まさ}るとは決して思っておらん。しかしいくら不器量の吾輩でも、今吾輩の主人に\ruby{描}{えが}き出されつつあるような妙な姿とは、どうしても思われない。第一色が違う。吾輩は\ruby{波斯産}{ペルシャさん}の猫のごとく黄を含める淡灰色に\ruby{漆}{うるし}のごとき\ruby{斑入}{ふい}りの皮膚を有している。これだけは誰が見ても疑うべからざる事実と思う。しかるに今主人の彩色を見ると、黄でもなければ黒でもない、灰色でもなければ\ruby{褐色}{とびいろ}でもない、さればとてこれらを交ぜた色でもない。ただ一種の色であるというよりほかに評し方のない色である。その上不思議な事は眼がない。もっともこれは寝ているところを写生したのだから無理もないが眼らしい所さえ見えないから\ruby{盲猫}{めくら}だか寝ている猫だか判然しないのである。吾輩は心中ひそかにいくらアンドレア・デル・サルトでもこれではしようがないと思った。しかしその熱心には感服せざるを得ない。なるべくなら動かずにおってやりたいと思ったが、さっきから小便が催うしている。\ruby{身内}{みうち}の筋肉はむずむずする。\ruby{最早}{もはや}一分も\ruby{猶予}{ゆうよ}が出来ぬ\ruby{仕儀}{しぎ}となったから、やむをえず失敬して両足を前へ存分のして、首を低く押し出してあーあと\ruby{大}{だい}なる欠伸をした。さてこうなって見ると、もうおとなしくしていても仕方がない。どうせ主人の予定は\ruby{打}{ぶ}ち\ruby{壊}{こ}わしたのだから、ついでに裏へ行って用を\ruby{足}{た}そうと思ってのそのそ這い出した。すると主人は失望と怒りを\ruby{掻}{か}き交ぜたような声をして、座敷の中から「この馬鹿野郎」と\ruby{怒鳴}{どな}った。この主人は人を\ruby{罵}{ののし}るときは必ず馬鹿野郎というのが癖である。ほかに悪口の言いようを知らないのだから仕方がないが、今まで辛棒した人の気も知らないで、\ruby{無暗}{むやみ}に馬鹿野郎\ruby{呼}{よば}わりは失敬だと思う。それも平生吾輩が彼の\ruby{背中}{せなか}へ乗る時に少しは好い顔でもするならこの\ruby{漫罵}{まんば}も甘んじて受けるが、こっちの便利になる事は何一つ快くしてくれた事もないのに、小便に立ったのを馬鹿野郎とは\ruby{酷}{ひど}い。元来人間というものは自己の力量に慢じてみんな増長している。少し人間より強いものが出て来て\ruby{窘}{いじ}めてやらなくてはこの先どこまで増長するか分らない。 \newline
 \ruby{我儘}{わがまま}もこのくらいなら我慢するが吾輩は人間の不徳についてこれよりも数倍悲しむべき報道を耳にした事がある。 \newline
 吾輩の家の裏に十坪ばかりの\ruby{茶園}{ちゃえん}がある。広くはないが\ruby{瀟洒}{さっぱり}とした心持ち好く日の\ruby{当}{あた}る所だ。うちの小供があまり騒いで楽々昼寝の出来ない時や、あまり退屈で腹加減のよくない折などは、吾輩はいつでもここへ出て\ruby{浩然}{こうぜん}の気を養うのが例である。ある小春の穏かな日の二時頃であったが、吾輩は\ruby{昼飯後}{ちゅうはんご}快よく一睡した\ruby{後}{のち}、運動かたがたこの茶園へと\ruby{歩}{ほ}を運ばした。茶の木の根を一本一本嗅ぎながら、西側の杉垣のそばまでくると、枯菊を押し倒してその上に大きな猫が前後不覚に寝ている。彼は吾輩の近づくのも\ruby{一向}{いっこう}心付かざるごとく、また心付くも無頓着なるごとく、大きな\ruby{鼾}{いびき}をして長々と体を\ruby{横}{よこた}えて眠っている。\ruby{他}{ひと}の庭内に忍び入りたるものがかくまで平気に\ruby{睡}{ねむ}られるものかと、吾輩は\ruby{窃}{ひそ}かにその大胆なる度胸に驚かざるを得なかった。彼は純粋の黒猫である。わずかに\ruby{午}{ご}を過ぎたる太陽は、透明なる光線を彼の皮膚の上に\ruby{抛}{な}げかけて、きらきらする\ruby{柔毛}{にこげ}の間より眼に見えぬ炎でも\ruby{燃}{も}え\ruby{出}{い}ずるように思われた。彼は猫中の大王とも云うべきほどの偉大なる体格を有している。吾輩の倍はたしかにある。吾輩は嘆賞の念と、好奇の心に前後を忘れて彼の前に\ruby{佇立}{ちょりつ}して余念もなく\ruby{眺}{なが}めていると、静かなる小春の風が、杉垣の上から出たる\ruby{梧桐}{ごとう}の枝を\ruby{軽}{かろ}く誘ってばらばらと二三枚の葉が枯菊の茂みに落ちた。大王はかっとその\ruby{真丸}{まんまる}の眼を開いた。今でも記憶している。その眼は人間の珍重する\ruby{琥珀}{こはく}というものよりも\ruby{遥}{はる}かに美しく輝いていた。彼は身動きもしない。\ruby{双眸}{そうぼう}の奥から射るごとき光を吾輩の\ruby{矮小}{わいしょう}なる\ruby{額}{ひたい}の上にあつめて、\ltjkenten{御めえ}は一体何だと云った。大王にしては少々言葉が\ruby{卑}{いや}しいと思ったが何しろその声の底に犬をも\ruby{挫}{ひ}しぐべき力が\ruby{籠}{こも}っているので吾輩は少なからず恐れを\ruby{抱}{いだ}いた。しかし\ruby{挨拶}{あいさつ}をしないと\ruby{険呑}{けんのん}だと思ったから「吾輩は猫である。名前はまだない」となるべく平気を\ruby{装}{よそお}って冷然と答えた。しかしこの時吾輩の心臓はたしかに平時よりも烈しく鼓動しておった。彼は\ruby{大}{おおい}に\ruby{軽蔑}{けいべつ}せる調子で「何、猫だ? 猫が聞いてあきれらあ。\ruby{全}{ぜん}てえどこに住んでるんだ」随分\ruby{傍若無人}{ぼうじゃくぶじん}である。「吾輩はここの教師の\ruby{家}{うち}にいるのだ」「どうせそんな事だろうと思った。いやに\ruby{瘠}{や}せてるじゃねえか」と大王だけに\ruby{気焔}{きえん}を吹きかける。言葉付から察するとどうも良家の猫とも思われない。しかしその\ruby{膏切}{あぶらぎ}って肥満しているところを見ると御馳走を食ってるらしい、豊かに暮しているらしい。吾輩は「そう云う君は一体誰だい」と聞かざるを得なかった。「\ruby{己}{お}れあ車屋の\ruby{黒}{くろ}よ」\ruby{昂然}{こうぜん}たるものだ。車屋の黒はこの近辺で知らぬ者なき乱暴猫である。しかし車屋だけに強いばかりでちっとも教育がないからあまり誰も交際しない。同盟敬遠主義の\ruby{的}{まと}になっている奴だ。吾輩は彼の名を聞いて少々尻こそばゆき感じを起すと同時に、一方では少々\ruby{軽侮}{けいぶ}の念も生じたのである。吾輩はまず彼がどのくらい無学であるかを\ruby{試}{ため}してみようと思って\ruby{左}{さ}の問答をして見た。 \newline
「一体車屋と教師とはどっちがえらいだろう」 \newline
「車屋の方が強いに\ruby{極}{きま}っていらあな。\ltjkenten{御めえ}の\ltjkenten{うち}の主人を見ねえ、まるで骨と皮ばかりだぜ」 \newline
「君も車屋の猫だけに\ruby{大分}{だいぶ}強そうだ。車屋にいると\ruby{御馳走}{ごちそう}が食えると見えるね」 \newline
「\ruby{何}{なあ}に\ltjkenten{おれ}なんざ、どこの国へ行ったって食い物に不自由はしねえつもりだ。\ltjkenten{御めえ}なんかも\ruby{茶畠}{ちゃばたけ}ばかりぐるぐる廻っていねえで、ちっと\ruby{己}{おれ}の\ruby{後}{あと}へくっ付いて来て見ねえ。一と月とたたねえうちに見違えるように太れるぜ」 \newline
「追ってそう願う事にしよう。しかし\ruby{家}{うち}は教師の方が車屋より大きいのに住んでいるように思われる」 \newline
「\ruby{箆棒}{べらぼう}め、うちなんかいくら大きくたって腹の\ruby{足}{た}しになるもんか」 \newline
 彼は\ruby{大}{おおい}に\ruby{肝癪}{かんしゃく}に\ruby{障}{さわ}った様子で、\ruby{寒竹}{かんちく}をそいだような耳をしきりとぴく付かせてあららかに立ち去った。吾輩が車屋の黒と\ruby{知己}{ちき}になったのはこれからである。 \newline
 その\ruby{後}{ご}吾輩は\ruby{度々}{たびたび}黒と\ruby{邂逅}{かいこう}する。邂逅する\ruby{毎}{ごと}に彼は車屋相当の\ruby{気焔}{きえん}を吐く。先に吾輩が耳にしたという不徳事件も実は黒から聞いたのである。 \newline
 或る日例のごとく吾輩と黒は暖かい\ruby{茶畠}{ちゃばたけ}の中で\ruby{寝転}{ねころ}びながらいろいろ雑談をしていると、彼はいつもの\ruby{自慢話}{じまんばな}しをさも新しそうに繰り返したあとで、吾輩に向って\ruby{下}{しも}のごとく質問した。「\ltjkenten{御めえ}は今までに鼠を何匹とった事がある」智識は黒よりも余程発達しているつもりだが腕力と勇気とに至っては\ruby{到底}{とうてい}黒の比較にはならないと覚悟はしていたものの、この問に接したる時は、さすがに\ruby{極}{きま}りが\ruby{善}{よ}くはなかった。けれども事実は事実で\ruby{詐}{いつわ}る訳には行かないから、吾輩は「実はとろうとろうと思ってまだ\ruby{捕}{と}らない」と答えた。黒は彼の鼻の先からぴんと\ruby{突張}{つっぱ}っている長い\ruby{髭}{ひげ}をびりびりと\ruby{震}{ふる}わせて非常に笑った。元来黒は自慢をする\ruby{丈}{だけ}にどこか足りないところがあって、彼の\ruby{気焔}{きえん}を感心したように\ruby{咽喉}{のど}をころころ鳴らして謹聴していればはなはだ\ruby{御}{ぎょ}しやすい猫である。吾輩は彼と近付になってから\ruby{直}{すぐ}にこの呼吸を飲み込んだからこの場合にもなまじい\ruby{己}{おの}れを弁護してますます形勢をわるくするのも\ruby{愚}{ぐ}である、いっその事彼に自分の手柄話をしゃべらして御茶を濁すに\ruby{若}{し}くはないと思案を\ruby{定}{さだ}めた。そこでおとなしく「君などは年が年であるから\ruby{大分}{だいぶん}とったろう」とそそのかして見た。果然彼は\ruby{墻壁}{しょうへき}の\ruby{欠所}{けっしょ}に\ruby{吶喊}{とっかん}して来た。「たんとでもねえが三四十はとったろう」とは得意気なる彼の答であった。彼はなお語をつづけて「鼠の百や二百は一人でいつでも引き受けるが\ltjkenten{いたち}ってえ奴は手に合わねえ。一度\ltjkenten{いたち}に向って\ruby{酷}{ひど}い目に\ruby{逢}{あ}った」「へえなるほど」と\ruby{相槌}{あいづち}を打つ。黒は大きな眼をぱちつかせて云う。「去年の大掃除の時だ。うちの亭主が\ruby{石灰}{いしばい}の袋を持って\ruby{椽}{えん}の下へ\ruby{這}{は}い込んだら\ltjkenten{御めえ}大きな\ltjkenten{いたち}の野郎が\ruby{面喰}{めんくら}って飛び出したと思いねえ」「ふん」と感心して見せる。「\ltjkenten{いたち}ってけども何鼠の少し大きいぐれえのものだ。こん\ruby{畜生}{ちきしょう}って気で追っかけてとうとう\ruby{泥溝}{どぶ}の中へ追い込んだと思いねえ」「うまくやったね」と\ruby{喝采}{かっさい}してやる。「ところが\ltjkenten{御めえ}いざってえ段になると奴め\ruby{最後}{さいご}っ\ruby{屁}{ぺ}をこきゃがった。\ruby{臭}{くせ}えの臭くねえのってそれからってえものは\ltjkenten{いたち}を見ると胸が悪くならあ」彼はここに至ってあたかも去年の臭気を\ruby{今}{いま}なお感ずるごとく前足を揚げて鼻の頭を二三遍なで廻わした。吾輩も少々気の毒な感じがする。ちっと景気を付けてやろうと思って「しかし鼠なら君に\ruby{睨}{にら}まれては百年目だろう。君はあまり鼠を\ruby{捕}{と}るのが名人で鼠ばかり食うものだからそんなに肥って色つやが善いのだろう」黒の御機嫌をとるためのこの質問は不思議にも反対の結果を\ruby{呈出}{ていしゅつ}した。彼は\ruby{喟然}{きぜん}として\ruby{大息}{たいそく}していう。「\ruby{考}{かん}げえるとつまらねえ。いくら稼いで鼠をとったって――一てえ人間ほどふてえ奴は世の中にいねえぜ。人のとった鼠をみんな取り上げやがって交番へ持って行きゃあがる。交番じゃ誰が\ruby{捕}{と}ったか分らねえからその\ltjkenten{たんび}に五銭ずつくれるじゃねえか。うちの亭主なんか\ruby{己}{おれ}の御蔭でもう壱円五十銭くらい\ruby{儲}{もう}けていやがる癖に、\ruby{碌}{ろく}なものを食わせた事もありゃしねえ。おい人間てものあ\ruby{体}{てい}の\ruby{善}{い}い泥棒だぜ」さすが無学の黒もこのくらいの\ruby{理窟}{りくつ}はわかると見えてすこぶる\ruby{怒}{おこ}った\ruby{容子}{ようす}で背中の毛を\ruby{逆立}{さかだ}てている。吾輩は少々気味が悪くなったから善い加減にその場を\ruby{胡魔化}{ごまか}して\ruby{家}{うち}へ帰った。この時から吾輩は決して鼠をとるまいと決心した。しかし黒の子分になって鼠以外の御馳走を\ruby{猟}{あさ}ってあるく事もしなかった。御馳走を食うよりも寝ていた方が気楽でいい。教師の\ruby{家}{うち}にいると猫も教師のような性質になると見える。要心しないと今に胃弱になるかも知れない。 \newline
 教師といえば吾輩の主人も近頃に至っては\ruby{到底}{とうてい}水彩画において\ruby{望}{のぞみ}のない事を悟ったものと見えて十二月一日の日記にこんな事をかきつけた。 \newline

\begin{quotation}
○○と云う人に今日の会で始めて\ruby{出逢}{であ}った。あの人は\ruby{大分}{だいぶ}\ruby{放蕩}{ほうとう}をした人だと云うがなるほど\ruby{通人}{つうじん}らしい\ruby{風采}{ふうさい}をしている。こう云う\ruby{質}{たち}の人は女に好かれるものだから○○が放蕩をしたと云うよりも放蕩をするべく余儀なくせられたと云うのが適当であろう。あの人の妻君は芸者だそうだ、\ruby{羨}{うらや}ましい事である。元来放蕩家を悪くいう人の大部分は放蕩をする資格のないものが多い。また放蕩家をもって自任する連中のうちにも、放蕩する資格のないものが多い。これらは余儀なくされないのに無理に進んでやるのである。あたかも吾輩の水彩画に於けるがごときもので到底卒業する気づかいはない。しかるにも関せず、自分だけは通人だと思って\ruby{済}{すま}している。料理屋の酒を飲んだり待合へ\ruby{這入}{はい}るから通人となり得るという論が立つなら、吾輩も\ruby{一廉}{ひとかど}の水彩画家になり得る\ruby{理窟}{りくつ}だ。吾輩の水彩画のごときはかかない方がましであると同じように、\ruby{愚昧}{ぐまい}なる通人よりも山出しの\ruby{大野暮}{おおやぼ}の方が\ruby{遥}{はる}かに上等だ。
\end{quotation}

 \ruby{通人論}{つうじんろん}はちょっと\ruby{首肯}{しゅこう}しかねる。また芸者の妻君を羨しいなどというところは教師としては口にすべからざる愚劣の考であるが、自己の水彩画における批評眼だけはたしかなものだ。主人はかくのごとく\ruby{自知}{じち}の\ruby{明}{めい}あるにも関せずその\ruby{自惚心}{うぬぼれしん}はなかなか抜けない。\ruby{中二日}{なかふつか}置いて十二月四日の日記にこんな事を書いている。 \newline

\begin{quotation}
\ruby{昨夜}{ゆうべ}は僕が水彩画をかいて到底物にならんと思って、そこらに\ruby{抛}{ほう}って置いたのを誰かが立派な額にして\ruby{欄間}{らんま}に\ruby{懸}{か}けてくれた夢を見た。さて額になったところを見ると我ながら急に上手になった。非常に嬉しい。これなら立派なものだと\ruby{独}{ひと}りで眺め暮らしていると、夜が明けて眼が\ruby{覚}{さ}めてやはり元の通り下手である事が朝日と共に明瞭になってしまった。 \newline
\end{quotation}

 主人は夢の\ruby{裡}{うち}まで水彩画の未練を\ruby{背負}{しょ}ってあるいていると見える。これでは水彩画家は無論\ruby{夫子}{ふうし}の\ruby{所謂}{いわゆる}通人にもなれない\ruby{質}{たち}だ。 \newline
 主人が水彩画を夢に見た翌日例の金縁\ruby{眼鏡}{めがね}の美学者が久し振りで主人を訪問した。彼は座につくと\ruby{劈頭}{へきとう}第一に「\ruby{画}{え}はどうかね」と口を切った。主人は平気な顔をして「君の忠告に従って写生を\ruby{力}{つと}めているが、なるほど写生をすると今まで気のつかなかった物の形や、色の精細な変化などがよく分るようだ。西洋では\ruby{昔}{むか}しから写生を主張した結果\ruby{今日}{こんにち}のように発達したものと思われる。さすがアンドレア・デル・サルトだ」と日記の事は\ltjkenten{おくび}にも出さないで、またアンドレア・デル・サルトに感心する。美学者は笑いながら「実は君、あれは\ruby{出鱈目}{でたらめ}だよ」と頭を\ruby{掻}{か}く。「何が」と主人はまだ\ruby{譃}{いつ}わられた事に気がつかない。「何がって君のしきりに感服しているアンドレア・デル・サルトさ。あれは僕のちょっと\ruby{捏造}{ねつぞう}した話だ。君がそんなに\ruby{真面目}{まじめ}に信じようとは思わなかったハハハハ」と大喜悦の\ruby{体}{てい}である。吾輩は椽側でこの対話を聞いて彼の今日の日記にはいかなる事が\ruby{記}{しる}さるるであろうかと\ruby{予}{あらかじ}め想像せざるを得なかった。この美学者はこんな\ruby{好}{いい}加減な事を吹き散らして人を\ruby{担}{かつ}ぐのを唯一の\ruby{楽}{たのしみ}にしている男である。彼はアンドレア・デル・サルト事件が主人の\ruby{情線}{じょうせん}にいかなる響を伝えたかを\ruby{毫}{ごう}も顧慮せざるもののごとく得意になって\ruby{下}{しも}のような事を\ruby{饒舌}{しゃべ}った。「いや時々\ruby{冗談}{じょうだん}を言うと人が\ruby{真}{ま}に受けるので\ruby{大}{おおい}に\ruby{滑稽的}{こっけいてき}美感を\ruby{挑撥}{ちょうはつ}するのは面白い。せんだってある学生にニコラス・ニックルベーがギボンに忠告して彼の一世の大著述なる仏国革命史を仏語で書くのをやめにして英文で出版させたと言ったら、その学生がまた馬鹿に記憶の善い男で、日本文学会の演説会で真面目に僕の話した通りを繰り返したのは滑稽であった。ところがその時の傍聴者は約百名ばかりであったが、皆熱心にそれを傾聴しておった。それからまだ面白い話がある。せんだって或る文学者のいる席でハリソンの歴史小説セオファーノの\ruby{話}{はな}しが出たから僕はあれは歴史小説の\ruby{中}{うち}で\ruby{白眉}{はくび}である。ことに女主人公が死ぬところは\ruby{鬼気}{きき}人を襲うようだと評したら、僕の向うに坐っている知らんと云った事のない先生が、そうそうあすこは実に名文だといった。それで僕はこの男もやはり僕同様この小説を読んでおらないという事を知った」神経胃弱性の主人は眼を丸くして問いかけた。「そんな\ruby{出鱈目}{でたらめ}をいってもし相手が読んでいたらどうするつもりだ」あたかも人を\ruby{欺}{あざむ}くのは\ruby{差支}{さしつかえ}ない、ただ\ruby{化}{ばけ}の\ruby{皮}{かわ}があらわれた時は困るじゃないかと感じたもののごとくである。美学者は少しも動じない。「なにその\ruby{時}{とき}ゃ別の本と間違えたとか何とか云うばかりさ」と云ってけらけら笑っている。この美学者は金縁の眼鏡は掛けているがその性質が車屋の黒に似たところがある。主人は黙って日の出を輪に吹いて吾輩にはそんな勇気はないと云わんばかりの顔をしている。美学者はそれだから\ruby{画}{え}をかいても駄目だという目付で「しかし\ruby{冗談}{じょうだん}は冗談だが画というものは実際むずかしいものだよ、レオナルド・ダ・ヴィンチは門下生に寺院の壁の\ltjkenten{しみ}を写せと教えた事があるそうだ。なるほど\ruby{雪隠}{せついん}などに\ruby{這入}{はい}って雨の漏る壁を余念なく眺めていると、なかなかうまい模様画が自然に出来ているぜ。君注意して写生して見給えきっと面白いものが出来るから」「また\ruby{欺}{だま}すのだろう」「いえこれだけはたしかだよ。実際奇警な語じゃないか、ダ・ヴィンチでもいいそうな事だあね」「なるほど奇警には相違ないな」と主人は半分降参をした。しかし彼はまだ雪隠で写生はせぬようだ。 \newline
 車屋の黒はその\ruby{後}{ご}\ruby{跛}{びっこ}になった。彼の光沢ある毛は\ruby{漸々}{だんだん}色が\ruby{褪}{さ}めて抜けて来る。吾輩が\ruby{琥珀}{こはく}よりも美しいと評した彼の眼には\ruby{眼脂}{めやに}が一杯たまっている。ことに著るしく吾輩の注意を\ruby{惹}{ひ}いたのは彼の元気の消沈とその体格の悪くなった事である。吾輩が例の\ruby{茶園}{ちゃえん}で彼に逢った最後の日、どうだと云って尋ねたら「\ltjkenten{いたち}の\ruby{最後屁}{さいごっぺ}と\ruby{肴屋}{さかなや}の\ruby{天秤棒}{てんびんぼう}には\ruby{懲々}{こりごり}だ」といった。 \newline
 赤松の間に二三段の\ruby{紅}{こう}を綴った\ruby{紅葉}{こうよう}は\ruby{昔}{むか}しの夢のごとく散って\ltjkenten{つくばい}に近く代る代る\ruby{花弁}{はなびら}をこぼした\ruby{紅白}{こうはく}の\ruby{山茶花}{さざんか}も残りなく落ち尽した。三間半の南向の椽側に冬の日脚が早く傾いて\ruby{木枯}{こがらし}の吹かない日はほとんど\ruby{稀}{まれ}になってから吾輩の昼寝の時間も\ruby{狭}{せば}められたような気がする。 \newline
 主人は毎日学校へ行く。帰ると書斎へ立て\ruby{籠}{こも}る。人が来ると、教師が\ruby{厭}{いや}だ厭だという。水彩画も滅多にかかない。タカジヤスターゼも功能がないといってやめてしまった。小供は感心に休まないで幼稚園へかよう。帰ると唱歌を歌って、\ruby{毬}{まり}をついて、時々吾輩を\ruby{尻尾}{しっぽ}でぶら下げる。 \newline
 吾輩は\ruby{御馳走}{ごちそう}も食わないから別段\ruby{肥}{ふと}りもしないが、まずまず健康で\ruby{跛}{びっこ}にもならずにその日その日を暮している。鼠は決して取らない。おさんは\ruby{未}{いま}だに\ruby{嫌}{きら}いである。名前はまだつけてくれないが、欲をいっても際限がないから\ruby{生涯}{しょうがい}この教師の\ruby{家}{うち}で無名の猫で終るつもりだ。 \newline
\newline

\end{document}